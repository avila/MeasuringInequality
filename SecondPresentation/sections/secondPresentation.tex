\PassOptionsToPackage{unicode=true}{hyperref} % options for packages loaded elsewhere
\PassOptionsToPackage{hyphens}{url}
%
\documentclass[ignorenonframetext,]{beamer}
\usepackage{pgfpages}
\setbeamertemplate{caption}[numbered]
\setbeamertemplate{caption label separator}{: }
\setbeamercolor{caption name}{fg=normal text.fg}
\beamertemplatenavigationsymbolsempty
% Prevent slide breaks in the middle of a paragraph:
\widowpenalties 1 10000
\raggedbottom
\setbeamertemplate{part page}{
\centering
\begin{beamercolorbox}[sep=16pt,center]{part title}
  \usebeamerfont{part title}\insertpart\par
\end{beamercolorbox}
}
\setbeamertemplate{section page}{
\centering
\begin{beamercolorbox}[sep=12pt,center]{part title}
  \usebeamerfont{section title}\insertsection\par
\end{beamercolorbox}
}
\setbeamertemplate{subsection page}{
\centering
\begin{beamercolorbox}[sep=8pt,center]{part title}
  \usebeamerfont{subsection title}\insertsubsection\par
\end{beamercolorbox}
}
\AtBeginPart{
  \frame{\partpage}
}
\AtBeginSection{
  \ifbibliography
  \else
    \frame{\sectionpage}
  \fi
}
\AtBeginSubsection{
  \frame{\subsectionpage}
}
\usepackage{lmodern}
\usepackage{amssymb,amsmath}
\usepackage{ifxetex,ifluatex}
\usepackage{fixltx2e} % provides \textsubscript
\ifnum 0\ifxetex 1\fi\ifluatex 1\fi=0 % if pdftex
  \usepackage[T1]{fontenc}
  \usepackage[utf8]{inputenc}
  \usepackage{textcomp} % provides euro and other symbols
\else % if luatex or xelatex
  \usepackage{unicode-math}
  \defaultfontfeatures{Ligatures=TeX,Scale=MatchLowercase}
\fi
\usetheme[]{Antibes}
\usecolortheme{dolphin}
\usefonttheme{professionalfonts}
% use upquote if available, for straight quotes in verbatim environments
\IfFileExists{upquote.sty}{\usepackage{upquote}}{}
% use microtype if available
\IfFileExists{microtype.sty}{%
\usepackage[]{microtype}
\UseMicrotypeSet[protrusion]{basicmath} % disable protrusion for tt fonts
}{}
\IfFileExists{parskip.sty}{%
\usepackage{parskip}
}{% else
\setlength{\parindent}{0pt}
\setlength{\parskip}{6pt plus 2pt minus 1pt}
}
\usepackage{hyperref}
\hypersetup{
            pdfborder={0 0 0},
            breaklinks=true}
\urlstyle{same}  % don't use monospace font for urls
\newif\ifbibliography
\usepackage{longtable,booktabs}
\usepackage{caption}
% These lines are needed to make table captions work with longtable:
\makeatletter
\def\fnum@table{\tablename~\thetable}
\makeatother
\setlength{\emergencystretch}{3em}  % prevent overfull lines
\providecommand{\tightlist}{%
  \setlength{\itemsep}{0pt}\setlength{\parskip}{0pt}}
\setcounter{secnumdepth}{0}

% set default figure placement to htbp
\makeatletter
\def\fps@figure{htbp}
\makeatother

\usepackage{adjustbox}
\usepackage{fontspec}
\useoutertheme{infolines}
\useinnertheme{circles}
\setbeamercolor{note page}{bg=white,fg=black}
\setbeamercolor{note title}{bg=white!80!black, fg=black}

\date{}

\begin{document}

\hypertarget{do-not-inlcude-this}{%
\section{DO NOT INLCUDE THIS}\label{do-not-inlcude-this}}

TODO: dont include it

\hypertarget{include-this}{%
\section{INCLUDE THIS}\label{include-this}}

\hypertarget{hausman-test}{%
\subsection{Hausman Test}\label{hausman-test}}

\begin{frame}[allowframebreaks]{Robustness Check of Pareto's alpha}
\protect\hypertarget{robustness-check-of-paretos-alpha}{}

Question: Is the Pareto distribution a \emph{good enough} approximation
of the wealth distribution among Germany's top wealth individuals?

If this is the case, one can expect that:

\begin{itemize}
\tightlist
\item
  the \(\alpha\) parameter is across SOEP and Pretest datasets and
\item
  among the chosen lower bounds, while taking the uncertainty from the
  imputation method into account.
\end{itemize}

To check this hypothesis we perfom a generalized Hausman Test based on
Hausman (1978) and Hausman and McFadden (1984).

\end{frame}

\begin{frame}[fragile]{Generalized Hausman Test}
\protect\hypertarget{generalized-hausman-test}{}

We test the stability of the estimated Pareto's \(\alpha\) by cross
checking the obtained coefficients among a combination of each dataset,
percentile and imputed data, via stata's \texttt{suest} post-estimation
command.

\[H_0: \hat\alpha^{data_i}_{pct_k, imp_m} = \hat\alpha^{data_j}_{pct_l, imp_m}\]

where \(\textit{data}~in~\{SOEP, Pretest\} \text{ datasets}\),
\(\textit{pct}~in~\{95th, 99th\} \text{ percentiles}\) and
\(\textit{imp}~in~\{1, ... 5\} \text{ imputations}\).

The Hausman test statistic follows

\[ \frac{  \left( \hat\alpha_i - \hat\alpha_j \right)^2 }
    {var(\hat\alpha_i) - 2  cov(\hat\alpha_i,\hat\alpha_j) 
      + var(\hat\alpha_j)} \sim \chi^2_{1}.
\]

\note{\begin{itemize}\tightlist

\item we use Hausman via suest because we are able to compare coefficients 
of two distinct regressions and grouping the data would be problematic due to
undefined reweighting scheme for combined dataset

\end{itemize}}

\end{frame}

\hypertarget{test-results}{%
\subsection{Test Results}\label{test-results}}

\begin{frame}{Results I}
\protect\hypertarget{results-i}{}

\begin{table}[]
  \centering
  \resizebox{\textwidth}{!}{\begin{table} \centering \begin{tabular}{ccccccc}
\hline \textit{} &   \textit{$\alpha_{SOEP}$} &      \textit{{SOEP}$} &   \textit{$\alpha_{Pretest}$} &   \textit{{Pretest}$} & \textit{threshold} &   \textit{p-value} \\ \hline
ln(net wealth 1&1.899&1809788&.853&305564&563800&\\
&(.0001)&&(.0009)&&&\\
net wealth 2&1.945&1807154&.851&309088&563570&\\
&(.0001)&&(.0009)&&&\\
net wealth 3&1.899&1809041&.907&306460&571000&\\
&(.0001)&&(.0009)&&&\\
net wealth 4&1.898&1809887&.858&312446&573000&\\
&(.0002)&&(.0009)&&&\\
net wealth 5&1.919&1809395&.841&306794&572000&\\
&(.0001)&&(.0009)&&&\\
mean of alphas&1.912&&.862&&&0\\
&(.002)&&(.0032)&&&\\
CI lower (.025)&1.824&&.751&&&\\
CI upper (.975)&1.999&&.973&&&\\
\hline \multicolumn{7}{l}{% \begin{minipage}{10cm}%  \vspace{.1cm} Source: SOEP (2012, v33.1) and Pretest 2017. \\Note: p-value represents the test result of xxx nw: net wealth, sd in parenthesis, household weights applied to SOEP, own calculated weights applied to the Pretest \end{minipage}% } \\  \end{tabular} \caption{Estimation Results of Pareto's Alpha (threshold at p95)} \label{tab:reference} \end{table}
}
  \caption{Estimation of Pareto's $\alpha$ across data and imputated sets
  for the lower bound at the 95th percentile.}
    \label{tab:my_label}
\end{table}

\end{frame}

\begin{frame}{Results II}
\protect\hypertarget{results-ii}{}

\begin{table}[]
        \centering
        \resizebox{\textwidth}{!}{\begin{tabular}{ccccccccc}
\hline \textit{}&   \textit{alpha\_SOEP}&   \textit{N\_SOEP}&       \textit{alpha\_Pretest}&        \textit{N\_Pretest}&    textit{threshold}&      \textit{Hausman (p-value)}\\ \hline
net wealth 1&-2.818&204&-1.61&80&870000&.287\\
&(.17)&&(.28)&&&\\
net wealth 2&-2.894&205&-1.714&79&879250&.311\\
&(.16)&&(.29)&&&\\
net wealth 3&-2.645&204&-2.027&76&898000&.618\\
&(.2)&&(.32)&&&\\
net wealth 4&-2.92&204&-1.777&77&901250&.351\\
&(.19)&&(.3)&&&\\
net wealth 5&-2.911&203&-1.647&77&883999&.276\\
&(.18)&&(.29)&&&\\
CI lower (.025)&-3.333&&-2.465&&&\\
CI upper (.975)&-2.343&&-1.045&&&\\
\hline \end{tabular}
}
        \caption{Estimation of Pareto's $\alpha$ across data and imputated sets for the 
        lower bound at the 99th percentile.}
        \label{tab:my_label}
\end{table}

\end{frame}

\end{document}
